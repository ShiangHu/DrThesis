\chapter{全文总结与展望}
\section{全文总结}
对于大样本数据的质量控制,我们从物理学、数学和统计学角度分析认为零参考优于其他参考,应当作为数据标准化结合BIDS使用;
我们提出了对去伪差程度的质量控制准则PaLOS指数,可用于筛选大量数据。
我们分析了多国家脑电数据,建立了独立与国家和个体因素的常模曲面,肯定了前人在某地区被试个体的常模曲线。
我们提出了谱节律成分提取的非参数拟合算法“ε节律”,具有较强的理论性。

\section{后续工作展望}
定量脑电技术与分析方法的研究近几年发展迅速,在本文研究工作的基础上,仍有以下方向值得进一步研究:
\begin{itemize}
	\item 多国家脑电数据集样本量、年龄不匹配,我们收集更多数据建立国际统一的常模。
	\item 进行神经源空间的常模分析。
	\item 利用协方差统计学,进行黎曼学习得到交叉谱基于的常模、分类诊断等。
	\item EM算法的正则化参数选择,初始化,收敛速度,有望通过主要最小化(MM)算法得到改进。
	\item 提出的新方法有待于在更多的实验数据中得到验证和改进。
\end{itemize}