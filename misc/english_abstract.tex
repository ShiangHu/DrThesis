
\begin{englishabstract}
Quantitative EEG (qEEG) is a diagnostic method based on the spectra norm of resting state EEG. With the large samples of EEG, the spectra norm varying with aging is the brain developmental trajectory, which can be a referable standard of disorders during brain developing, maturation and aging. The neural oscillation of different rhythms are shown as peaks at different frequency bands in the EEG spectra norm. The split-extraction of spectral peaks is a important method to qEEG analysis. In this thesis, as the order of fundamental, essential and important effects to qEEG study, or simple volume conduction model, spectra regression model and advanced oscillatory rhythms fitting model, we gradually investigated the reference reference problem and EEG spectra quality control issue, analyzed the possibility of creating the multinational qEEG norm with multinational EEG dataset, and developed a novel procedure to quantify the spectral oscillation. The main contents are as follows:

1. The average reference (AR) and the reference electrode standardization technique (REST) are two common references but competing options in qEEG analysis. Their differences under actual recording conditions are yet clear and we have no deep understanding to their theoretical foundation. Based on their physical assumptions, we simulated the gold standard -- the EEG potentials with infinity reference, comprehensively analyzed how different physical factors (11 channel numbers, 2 electrode layouts, 2 head shapes, the position of dipole sources, the scalp region, the perturbation to the volume conduction model, and the sensor noise, etc.) affect the performances of references on approximating the EEG potentials with infinity reference. We proposed the general reference model to unify various of reference estimators and obtained the unified reference estimator by the means of the penalized maximum likelihood estimate and generalized inverse. It was found that AR and REST were two particular cases of the unified estimator but differing at the spatial priors. With the lemma of inverting the full rank minus one modification type matrix, we demonstrated that REST was an unipolar reference so that the family of unipolar references was built after more cases were studied. We surprisingly derived the "no memory property: as the interrelations of unipolar references and the common properties of "full rank reduced by one" and "orthogonal projector weighted centering".

2. Although plenty of preprocessing pipelines can reject the artifacts of noisy EEG signal and get the good-looking waveforms, the methods lacking of quality metric such as correlation, interpolation, regression and independent component analysis may remove the brain activity during the artifacts correction, making the multichannel power spectra parallel (Parallel LOg Spectra (PaLOS)),which is called the issue of homogeneous spectra. Here, we performed the common principle analysis with the identical orthogonal basis for the cross spectra at all frequencies, and took the explained variance ratio of the 1st component as the PaLOS metric to the issue of homogeneous spectra. We calculated the PaLOS metric for 3 different sizes of example datasets and found that PaLOS metric can get the boxplots as the expected from the data quality. This metric can filter out some badly preprocessed or spectra-homogeneous cases in order to do qEEG analysis later.

3. The possibility of creating the multinational qEEG norm is still unclear, though there exists the multinational EEG microstates study or the qEEG norm study in one region. We analyzed 535 cases of EEG, whose age of being recorded almost covered the life span to investigate the effects of country and individual in creating the EEG norm with the linear mixed effects model. With the stepwise linear regression model and local weighted scatter smoothing non-parametric regression, we plotted the multinational qEEG developmental surface. It is first time to confirm that qEEG norm is independent of country, nationality, individual and culture.

4. About extracting the spectral rhythms, existing studies have proposed the likelihood ratio test or least residuals as the fitting criterion and student t or Gaussian kernel based parametric fitting, however, these criterion are not optimal to spectra estimation and the parametric fitting will hardly fit the rhythms with various shapes. Here, we started from the addition of Fourier coefficients of different components, and took the fitting of a rhythm as a problem of minimized Whittle likelihood。 Using the expectation maximization algorithm, we fitted various shapes of the spectra peaks with the penalization of smoothness and monotonicity in the maximization step. We analyzed the open iEEG dataset and extracted the features of spectra peaks. Based on the kriging method, we performed the interpolation of spectra features among volumetric dipoles and obtained the full brain oscillatory maps.

To summarize, we studied the physical factors of references, the information criterion based determination of references and the properties of unipolar references, which is the first time for us to investigate systematically the reference issue from physics and mathematical statistics; proposed the PaLOS metric based on the spectra homogeneity to filter the large sample of preprocessed EEG; confirmed the independency of qEEG norm to the effects of country and individual; developed the Whittle likelihood based nonparametric fitting algorithm to split rhythms and extract their features。 The mainly studied reference and spectra analysis techniques may play significant roles in qEEG analysis.

	\englishkeyword{quantitative EEG, REST, Average Reference, PaLOS, Norm, Spetral peaks}
\end{englishabstract}


