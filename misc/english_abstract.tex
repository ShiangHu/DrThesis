
\begin{englishabstract}
The EEG signals are the potential differences between the active electrodes and either the physical reference electrode or a virtual reference, presenting as the multichannel time series. The use of improper reference may distort the EEG waveform in the temporal domain, even affect the estimate of spectra, coherence and connectivity analysis.The debating issue of reference choice has become a primary question of quantitative EEG(qEEG). qEEG is a diagnostic method based on the normative features of spectra. The spectra related techniques determine the estimation of spectra norm, feature extraction and the diagnostic accuracy. Currently the lacking of quality control measure of spectra, multinational spectra norm and effective method on feature extraction constrains the application of qEEG in cognition and mental disorder related studies. This dissertation will focus on the reference choice and the spectra analysis. As the order of data quality and qEEG application, we studied the physical factors of references, the statistical evidence of reference determination and their mathematical properties in the first part, then proposed the quality measure of spectra, built the multinational norm and developed a method to extract the spectral features in the second part. The main contents are as follows:

1. There are large variety of references for use, including online recording references and re-references. They rely on the different physical assumptions but their performances during actual recording are unclear. By simulation, we obtained the standard EEG with infinity reference, then comprehensively analyzed the performances of 5 online recording references and 3 re-references on approximating the infinity reference w.r.t 11 channel numbers, 2 electrode layouts, head models, dipole position, scalp region, the perturbation of head model and sensor noise, etc. We found that average reference (AR) and zero reference (REST) are much better than others; AR relies on the coverage of sensors and the orientations of dipolar sources but has no close relation with channel number; REST is insensitive to the head model perturbation and shows the robustness under various factors. 

2. Some stimulative studies found that REST is superior to AR but their theoretical differences are unclear, lacking of the theoretical evidences for reference determination. We proposed a general reference model, obtained the unified Bayesian reference estimator by means of penalized maximum likelihood estimate and matrix inverse lemma, demonstrated AR and REST are two special cases of the Bayesian estimator under different spatial covariance priors where it is i.i.d for AR and volume conduction model based for REST. We derived the form of ridge regression for reference estimation that enabled the use of information criteria for reference choice. Both simulation and real data showed that REST has smaller information criteria than AR. We also studied the selection of regularization parameter and lead field matrix for REST. 

3. Unipolar reference is dominant in EEG reference. The relations between REST, AR and unipolar reference are yet clear. We have no understanding about their mathematical meaning and worry about if referencing multiple times will accumulate artifacts. By the full rank mines one type matrix inverse lemma, we demonstrated REST is a unipolar reference, built the family of unipolar references, and showed that REST and AR are the best estimators under their physical constraints. We derived the no memory, full rank deficient by 1, orthogonal projector centering properties which will accelerate the reference standardization. 

4. The use of ICA, interpolation and the other methods for artifact rejection may lose the brain activity related component, resulting into the issue of homogeneous spectra. We performed the common PCA for the cross spectra at all frequencies, then took the explained variance ratio of the $1^{st}$ component as the PaLOS metric to indicate the spectra homogeneity. In combination with visual inspection, we calculated the PaLOS metric for 3 databases with diffferent sizes and several steps of artifact rejection, and found that PaLOS can filter out
badly preprocessed or spectra-homogeneous cases in order to do qEEG analysis better. 

5. The studies of EEG norm are usually taken over isolated regions. The possibility of creating multinational norm is yet confirmed. We analyzed the EEG of 535 subjects from 3 countries, the age of which covered the whole lifespan. We studied the effects of country and individual by means of linear mixed effects model and created the multinational norm using stepwise linear regression and lowess. It was the first study to confirm that EEG spectra are not affected by country and individual. The multinational qEEG developmental surface replicates the previous results.

6. Based on the likelihood ratio or least square, the parametric fitting of student t or Gaussian curve hardly fits the spectra components with various shapes. We proposed the $\xi\pi$ model that assumed the Fourier coefficients of different components are additional and turned the estimation of spectra into the maximum Whittle likelihood estimate. This model can take the nonparametric fitting by constraining the smoothness and monotonicity and then extract the descriptive parameters. $\xi\pi$ model is superior to the parametric models. We applied this model to 1772 iEEG, extracted the features of spectra rhythms, and created the full brain oscillatory spectra atlas. 

To summarize, we have studied the physical factors of reference choice, the statistical evidence for reference determination and built the family of unipolar references. It is the first time to systematically analyze the reference problem from the perspective of physics and mathematical statistics; proposed the PaLOS metric for quality control of spectra, created the multinational qEEG norm and developed the $\xi\pi$ model to split spectral components and extract their features. The novel theories on reference and techniques of spectra analysis developed in this dissertation may be of great significance to updating the qEEG methods and pushing its applications in clinical studies.

	\englishkeyword{quantitative EEG, REST, Average reference, Spectral homogeneity, Spectral norm, Spectral components}
\end{englishabstract}