\begin{chineseabstract}
定量脑电是基于静息态脑电谱常模的诊断方法。 大样本下正常被试的脑电随着年龄变化的谱常模曲线可作为大脑正常的生长规律曲线,是研究神经功能发育成熟以及老化过程中出现异常的重要参照标准。在脑电的常模数据谱曲线中,不同节律成分的神经震荡活动呈现为谱峰,因此对
谱峰的分离和提取是定量脑电分析的一个重要方法。 在本论文中,我们按照对定量脑电分析具有根基性基础作用、本质关键作用、重要高级应用的次序,或者按照从简单的容积传导模型到谱特征回归模型再到更高阶的震荡节律成分拟合模型循序渐进,首先研究了质量控制环节中的参考标准化问题和脑电数据谱质量检验问题,随后我们分析了多国家脑电数据探究建立多国家之间定量脑电常模的可能性,最后我们发展了进行定量脑电谱曲线节律成分提取的新方法。 主要内容如下:

1. 平均参考和零参考是脑电参考中最常用但相互竞争的两个。 针对二者在实际采集条件下的性能区别尚不清楚,人们缺乏对两种参考的理论认识,我们从二者的物理学假设出发,通过仿真研究得到标准的无穷远参考下的脑电电位,综合分析了不同物理因素(包括11种不同数目的电极阵列、2种不同规律的电极分布、2种头模型、神经源的位置方向以及分区、头表电极区域分区、容积传导头模型的扰动、传感器电极噪声等)对参考变换标准的脑电电位前后误差的影响。我们提出了广义的线性参考模型,将不同类型的参考统一起来,采用惩罚的最大似然估计得到基于广义逆的最小模参考通解,发现平均参考和零参考是通解在不同协方差先验下的特例,采用广义交叉验证和贝叶斯信息准则对两种参考模型进行选
择。 随后,我们通过满秩减一型矩阵的广义逆引理,证明了零参考也是一种单极参考,建立了单极参考家族,并推导出了单极参考的“无记忆性”这种内在联系以及“满秩减一”和“正交投影加权中心化”共同属性。

2. 尽管各种各样的脑电预处理流程和分析平台能够把噪声较强的脑电数据处理得到漂亮的波形,但缺乏质量控制准则的预处理算法中基于相关、插值、回归或者独立成分分析的算法可能会丢失脑活动相关的信号,最终得到的多通道脑电功率谱相互平行(Parallel LOg Spectra (PaLOS)),这称为谱的同构问题。在本文中,我们对所有频率下的交叉谱进行具有共同正交基的主成分分析,将第一个主要成分的方差占比作为交叉谱同构异质的准则,称为PaLOS准则。 我们对三个不同样本大小的数据集以及分别预处理后的数据计算PaLOS准则,发现PaLOS能够得到与数据处
理前后质量变化一致的柱状图。 这种准则能够对大量、预处理后的脑电数据进行初步筛选去除一些去伪差过度或者谱同构的数据,便于进一步
定量分析。

3. 尽管存在多国家脑电微状态空间模式一致性或者某个地区脑电数谱常模的研究,多个国家间是否具有建立通用脑电常模的可能性尚不清楚。我们分析了来自3个不同国家基本覆盖整个生命年龄周期的535例脑电数据,采用线性混合效果模型研究了国家和个体对建立多国家间脑电谱常模的影响,使用分布线性回归模型和局部加权散点图平滑的非参数回归方法建立了多国家的脑电常模演化曲面,首次肯定了脑电谱常模不受国家、民族和文化等影响的特性。

4. 关于提取谱曲线的不同节律成分,以前的学者提出了基于似然比或最小化拟合残差准则的参数(铃形曲线或高斯核)的估计,这些方法都
不是基于最优谱估计的统计学准则,采用的参数拟合也难以对形态各异的节律成分较好地拟合。 本文中,我们从不同成分的频域信号傅里叶系数线性叠加出发,将某个节律成分谱的估计转换为最小化的Whittle似然问题。我们利用期望最大化算法,在最大化步骤采用基于平滑和单调性形态约束的方法可以对形态各异的谱成分进行拟合。 我们分析了公开的颅内脑电数据集,得到了脑电谱成分的节律特征。基于高斯过程回归,我们对不同节律特征进行体素间的插值,得到了覆盖全脑的震荡特征谱地形图。 

综上所述,我们研究了参考的物理因素、基于统计学习的参考选择、参考之间的内在联系和属性,首次从物理学、数学统计学意义上系统地对参考技术进行了分析;提出了基于谱异质同构特性的准则,可以用于对大样本脑电数据去伪差程度或者谱的质量进行初步筛选;分析了构建脑电谱常模的国家和个体影响因素,肯定了建立独立于国家因素定量脑电谱常模的可能性;发展了基于Whittle似然和非参数约束的$\xi$节律算法
能够有效地识别出谱曲线的多个成分并提取特征。 本文中重点研究的参考和谱分析相关技术对定量脑电分析具有重要意义。

\chinesekeyword{定量脑电,无穷远参考,零参考,谱分析,常模,节律成分}
\end{chineseabstract}