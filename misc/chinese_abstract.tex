\begin{chineseabstract}
脑电信号是活跃电极与物理参考电极或虚拟数字参考的电位差,呈现为多维时间序列。使用不当参考会造成时域波形失真甚至影响频域的谱、相干估计和网络分析等。参考选择不规范已成为定量脑电的首要问题。定量脑电是基于大样本静息态脑电谱常模特征的诊断方法,谱分析有关的技术如谱质量、常模估计、特征提取中存在的不足限制着定量脑电在神经认知功能和精神疾病方面的应用。本论文以参考选择和谱分析技术为重点,按照关乎数据质量、影响定量分析应用的顺序,第一部分深入研究参考选择的物理因素、统计学证据和单极参考的数学属性,第二部分初步探究谱质量准则、多国家谱常模估计和谱成分特征提取算法。
主要研究内容如下:

1. 当前供选择的参考种类繁多,多种在线记录参考和重参考虽具有不同物理假设,它们在实际采集条件下的性能区别尚不清楚。仿真得到标准无穷远参考下的脑电电位,综合分析5种在线记录参考和3种重参考在11种电极数、2种典型电极分布、多种头模型、不同神经源位置方向及分区、不同头表区域分区、不同头模型受扰动程度和不同电极噪声等条件下逼近理想无穷远参考的效果。研究发现平均参考和零参考明显更优,平均参考的性能并不随电极数增多而改善且受到头表电极覆盖程度和偶极子活动方向的制约,零参考对容积传导模型不敏感并在多种因素下表现出稳健性能。

2. 部分仿真研究发现零参考比平均参考更优,但二者理论差异并不清楚,还不存在能对二者进行选择的理论证据。将不同类型的参考统一为广义线性参考模型,采用惩罚最大似然估计并基于广义逆引理推导出参考的贝叶斯通解,证明平均参考和零参考是通解不同协方差先验时的特例,平均参考基于多通道脑电活动相互独立的先验,零参考则采用容积传导效应的先验。推导出参考估计的脊回归形式,采用广义交叉验证等信息准则对参考进行模型选择。仿真和实际数据分析均表明零参考比平均参考具有更优的模型指数。还对零参考中正则化参数和传递矩阵的选取进行了研究。

3. 单极参考是头表脑电参考的主要类型,我们并不清楚零参考、平均参考与单极参考的关系,缺少对零参考与平均参考数学意义的理解,对多次重参考是否会累积误差等存在疑惑。通过满秩减一型矩阵广义逆引理证明零参考是一种单极参考,建立单极参考家族,通过欠定的线性回归证明零参考和平均参考分别是对应物理约束下的最优估计,并推导出单极参考的无记忆性、满秩减一和正交投影加权中心化属性。统一的单极参考家族和属性有助于规范参考选择。

4. 独立成分分析等强力伪迹去除方法可能会丢失与神经活动有关的信号导致脑电谱同构问题。对所有频率的交叉谱矩阵进行具有共同正交基的主成分分析,将第一个主成分的方差占比作为表征交叉谱同构的PaLOS准则。结合视觉筛选,对三个样本量不同的数据库和多个伪迹去除步骤计算PaLOS准则,发现PaLOS能对预处理后的数据进行初步筛选去除一些去伪差过度或者谱同构的数据,便于进一步定量分析。

5. 虽然已存在某个地区脑电谱常模的研究,多个国家间是否能建立通用脑电常模尚不清楚。通过分析来自三个国家且基本覆盖整个生命周期的535例脑电数据,采用线性混合效应模型研究国家和个体对建立多国家间脑电谱常模的影响,使用分步线性回归模型和局部加权散点图平滑的非参数回归方法建立多国家脑电谱常模演化曲面,首次肯定脑电谱不受国家和个体的影响,得到与以前局部地区结果一致的多国家脑电谱常模演化曲面。

6. 基于似然比或最小二乘准则的铃形曲线或高斯核估计难以拟合形态各异的谱成分。我们提出$\xi\pi$模型,基于不同节律成分的傅里叶系数线性叠加将某个成分的谱估计转换为最大Whittle似然问题,先基于平滑和单调性约束对谱成分进行非参数拟合再提取参数。$\xi\pi$模型比参数拟合模型更优。
用$\xi\pi$模型分解并提取1772例颅内脑电数据谱成分的节律特征,基于高斯过程回归构建出覆盖全脑的振荡特征谱地形图。

综上,本文通过研究影响参考选择的物理因素并基于统计学习进行参考选择建立了单极参考家族,这是首次从物理、数理统计学视角对参考选择进行系统分析;提出能表征伪迹去除后脑电数据质量的谱同构准则,分析脑电谱常模的国家和个体因素并建立多国家脑电谱常模,发展了能有效分解谱节律成分并提取特征$\xi\pi$模型。本文发展的参考新理论和谱分析新技术可能对健全定量脑电分析方法,促进定量脑电的临床应用具有重要意义。

\chinesekeyword{定量脑电,零参考,平均参考,谱同构,谱常模,谱成分}
\end{chineseabstract}