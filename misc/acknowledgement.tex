\thesisacknowledgement
首先我要表达对导师Pedro A. Valdés-Sosa教授的感恩之情。教授知识渊博,富有科研激情,治学严谨,要求学生的科研工作必须具有原创性、科研成果具有说服力,注重培养学生兴趣和从根本上提升学生的科研素养,这些对基本功的培养使我受益良多;他为人率直、风趣幽默、关心学生、亦师亦友,一言一行深深影响了我。作为他的首届中国博士生,他悉心指导、言传身教、循循善诱,几年来如一日使我从懵懂科研门外汉成长为实验室的科研骨干。和他在午夜航班上讨论公式,在实验室小白板上讨论观点,在家里熬夜修改论文到天亮,远程电话里答疑解惑,视频里推导公式等都让我难以忘记。他为大众健康努力的科研理想和开放合作的科研态度深深影响
着我...

我要特别感谢合作导师尧德中教授,尧老师是我科研生涯的启蒙老师,为我指导迷津使我走出迷惘找到了科研兴趣;他学识渊博、富有前沿眼光、因材施教、耐心引导,他博学笃行、谦恭仁爱,在启迪我找到科研大门钥匙的同时还时常春风化雨般鼓励我向着正确方向努力。同时,我还要特别感谢联合实验室的Maria L. Brings-Vega、Jorge Bosch、Alan C. Evans、Thomas Koenig等教授,曾经或一直给予我关心指导的徐鹏教授、赖永秀、任鹏、郭大庆等副教授、卢竞、龚殿坤等老师。他们如父母兄长般的指导帮助让我感受到师恩似海,终生难忘。还要感谢在古巴神经科学中心和麦吉尔大学学习时的老师和同学们陪我度过了一段宝贵的访学时光。

同时,我也要感谢董烨芸老师在实验室科研生活中给予的很多帮助,感谢Esin Karahan、Eduardo G M、张锐、陈明明、王庆、李沛洋、李发礼等博士在科研方面的帮助,感谢李敏、Deirel Paz Linares、韩振峰、袁齐、郭艳博、王颖、黄孙培等同学的同窗陪伴。还有很多同学难以一一列举,在此一并表示感谢。

最后,我要感谢家人一直以来无私的爱,感恩家人的理解、支持和鼓励。