\thesisacknowledgement

% 在博士论文即将完成之际,我首先要表达对攻读博士学位期间导师Pedro A. Valdes-Sosa教授的感恩感激之情。 Pedro教授他知识渊博,富有科研激情,治学严谨,要求学生的科研工作必需具有创新性、科研成果具有说服力,注重培养学生的兴趣和从根本上提升学生的科研素养,这些对基本功的培养塑造了以后我科研道路的基础;他为人率直,风趣幽默,关心学生,亦师亦友,他的言行深深影响了我。作为他的首批中国博士生,他悉心指导,言传神教,循循善诱,几年来如一日,使我从懵懂科研门外汉成长为实验室的科研骨干。和他在半夜航班上讨论公式,在实验室小白板上讨论观点,在家里熬夜修改论文到天亮,远程电话里答疑解惑,视频里推导公式等等,这些都让我难以忘记。您为服务人类健康努力的科研理想和开放合作的科研态度深深影响了我...一日为师,终日为父,再多的感恩之情无法用文字表达,只希望将他的科研精神在以后的科研路上好好践行。

% 其次,我要特别感谢我的合作导师尧德中教授,尧老师是我科研生涯的启蒙老师,为我指导迷津,使我走出迷惘找到了科研兴趣;他学识渊博,富有前沿眼光,因材施教,耐心引导,他博学笃行,谦恭仁爱,在帮我找到科研大门的钥匙的同时也潜移默化地地提醒着我朝着正确的方向努力。同时,我还特别要感谢实验室的Maria L Brings Vega教授,曾经或一直给予我学业关心指导的徐鹏教授、赖永秀副教授、任鹏副教授、郭大庆副教授、卢竞老师、龚殿坤老师等。他们如父母兄长般指导帮助与教诲让我感受到师恩似海,终生难忘。还有学院的其他曾经给予我帮助的老师,我都对他们心存感激。

% 同时,我也要感谢董烨芸老师,她在实验室科研生活中给予了我很多帮助,感谢Esin Karahan博士,Eduardo G M博士,张锐博士、王庆博士、李沛洋博士、李发礼博士等在科研方面的帮助,感谢李敏、Deirel Paz Linares、韩振峰、袁齐、郭艳博、王颖、黄孙培等同学在学习和生活上的同窗陪伴。还有很多同学难以一一列举,在此一并表示感谢。

% 最后,我要感谢家人一直以来无私的爱与支持,她们是我求学路上的永远靠山和奋进不止努力求索的不竭精神动力源泉。